\documentclass{beamer}
\usepackage[utf8]{inputenc}
\usepackage{algorithmic}
\usepackage{algorithm}
\usepackage{amsfonts}
\usepackage{amssymb}
\usepackage{courier}
\usepackage{graphicx}
\usepackage{listings}
\usepackage{mathtools}
\usepackage[font={small}, labelfont={color=black}]{caption}
\usetheme{Rochester}

\usefonttheme[onlymath]{serif}
\beamertemplatenavigationsymbolsempty
\title{\LARGE{Evolving Behaviour Trees}\\
       \small{\emph{Learning Top--Down Shooter Behaviours}}}
\author{\vspace{2ex}\\\textbf{Martin Estgren} \;\,\;\;\,\,
        \texttt{<mares480@student.liu.se>} \\
        \textbf{Erik S. V. Jansson}\;
        \texttt{<erija578@student.liu.se>} \\~\\
        {Artificial Intelligence for Interactive Media}\\
        {ITN Linköping University (LiTH),\, Sweden}}

\lstset{
    escapeinside={<@}{@>},
    basicstyle=\tiny\ttfamily,
    breakatwhitespace = false,
    breaklines = true,
    captionpos = b,
    keepspaces = true,
    language = Java,
    showspaces = false,
    showstringspaces = false,
    frame = tb,
    aboveskip = 10pt,
    belowskip = 10pt,
    numbers = left,
    numbersep = 3pt
}

\setbeamerfont{bibliography item}{size=\small}
\setbeamerfont{bibliography entry author}{size=\small}
\setbeamerfont{bibliography entry title}{size=\small}
\setbeamerfont{bibliography entry location}{size=\small}
\setbeamerfont{bibliography entry note}{size=\small}
\setbeamertemplate{footline}[frame number]
\begin{document}
    \frame{\titlepage}

    \frame{\frametitle{Outline}
        \begin{enumerate}
            \item{Motivation}
            \item{Techniques}

                \begin{itemize}
                    \item{Behaviour Trees (BT)}
                    \item{Genetic Programming}
                \end{itemize}

            \item{Implementation}

            \item{Demo}

                \begin{itemize}
                    \item{Pre-Defined\, Behaviour Trees}
                    \item{GP Behaviour Tree Evolution}
                \end{itemize}

            \item{Results}

            \item{Discussion}
        \end{enumerate}
    }

    \frame{\frametitle{Motivation}
        \textbf{Goal:} create a fun and challenging adversarial bot to play against.\\
        For this we will need a couple of artificial intelligence ingredients:

        \vspace{1.25em}

        \begin{itemize}
            \item{Build game with a sufficiently complex environment: \emph{\underline{QUAKE}}}
            \item{A way to represent knowledge and decisions: \emph{\underline{Behaviour Trees}}}
            \item{Create interesting and novel behaviour:\ \emph{\underline{Genetic Programming}}}
        \end{itemize}
    }

    \frame{\frametitle{Techniques} \framesubtitle{Behaviour Trees}
        Models the decision-making process by using trees composed of \emph{control-flow nodes} and \emph{action/condition nodes}. States are either \emph{RUNNING}, \emph{FAILED}, \emph{SUCCEEDED} are updated every logic tick.

        \vspace{1em}

        \begin{figure}
            \centering
            \includegraphics[width=0.6\textwidth]{share/bt.png}
        \end{figure}
    }

    \frame{\frametitle{Techniques} \framesubtitle{Genetic Programming}
        \begin{enumerate}
            \item{\textbf{Selection:} technique for determining future \emph{populations} by picking \emph{naturally selected} individuals by an \emph{fitness function}.}
            \item{\textbf{Crossover:} pick \emph{individuals} in the population and produce their \emph{offspring} by mixing promising subset of their features.}
            \item{\textbf{Mutation:} \emph{alter} a individual's \emph{properties randomly} in order \\ to diversify the population: potentially get \emph{new behaviours}.}
        \end{enumerate}
        \begin{itemize}
            \item{\(\rightarrow\) very similar to the \emph{natural selection} process of \emph{evolution}.}
        \end{itemize}
    }

    \frame{\frametitle{Implementation}
        \begin{itemize}
            \item{\textbf{Behaviour trees:} we've implemented high-level actions e.g. attack ``player'' to limit the structural complexity of the tree.}
            \item{\textbf{A* path finding:} \emph{waypoint-based} navigation with \emph{Euclidean distance} as heuristic. Other player's field of view affects path cost (also called an \emph{influence map}): avoid player line of sight.}
            \item{\textbf{Genetic programming:} \emph{selection} of individuals proportional to \emph{fitness}. We \emph{mutate} the tree structure and node parameter randomly. We \emph{crossover} pairs of trees by swapping sub-trees.}
        \end{itemize}
    }

    \frame{\frametitle{Demo}
        \begin{itemize}
            \item{Pre-Defined\, Behaviour Trees}
                \begin{itemize}
                    \item{Demo Game}
                    \item{Demo Path Finding}
                    \item{Demo Behaviour Tree}
                    \item{Demo a Bot Match}
                \end{itemize}
            \item{GP Behaviour Tree Evolution}
        \end{itemize}
    }

    \frame{\frametitle{Results} \framesubtitle{Fitness by Generation}
        \begin{figure}[H]
            \includegraphics[width=\textwidth]{share/result2.png}
        \end{figure}
    }

    \frame{\frametitle{Results} \framesubtitle{Victories by Generation}
        \begin{figure}[H]
            \centering
            \includegraphics[height=\textheight]{share/result2-victories.png}
        \end{figure}
    }

    \frame{\frametitle{Results} \framesubtitle{Evolved Trees}
        \begin{figure}[H]
            \includegraphics[width=0.7\textwidth]{share/gpbt.png}
        \end{figure}
        \begin{figure}[H]
            \includegraphics[width=0.7\textwidth]{share/bt-gen.png}
        \end{figure}
    }

    \frame{\frametitle{Discussion}
        We've implemented a top-down shooter where bot decisions are made with \emph{behaviour trees}, controlled using \emph{A* path finding} and optimized using \emph{genetic programming}. All from scratch in Java.

        \begin{itemize}
            \item{\textbf{Our goals:} create a fun an challenging bot to play against.}
            \item{\textbf{What went right:} game is polished, behaviour trees allows to easily make complex behaviours, by genetic programming we create interesting new behaviours. C-E-S \& path finding.}
            \item{\textbf{What went wrong:} genetic programming didn't at first (we followed the paper, no metaheuristic). Many bugs related to physics.\; Maybe time-planned a bit wrong\, \(\rightarrow\)\, \textbf{crunch-time}.}
            \item{\textbf{Improvements:} analyze other selection heuristics, add more interesting actions, fix some adversarial resource competition.}
        \end{itemize}
    }

    \frame{\frametitle{Questions?}}
    \frame{\frametitle{Bibliography}
        \nocite{*}
        \bibliographystyle{alpha}
        \bibliography{slides}
    }
\end{document}
